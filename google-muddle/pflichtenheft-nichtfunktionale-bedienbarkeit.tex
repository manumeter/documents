\subsection{Bedienbarkeit}
\label{subsec:nichtfunktionale-bedienbarkeit}

Gemeint ist gemäss \cite{iso:9126} der \enquote{Aufwand für [...]
Benutzer[Innen], die Anwendung zu bedienen}.

\subsubsection{Beschreibung}

Da \textit{Google Muddle} vollständig automatisch funktionieren muss, ist keine
Interatktion mit BenutzerInnen gewünscht. Sie sollen auch nicht die Möglichkeit
haben, Konfigurationen vorzunehmen, da die Verhaltensweise der Applikation für
jedeN identisch sein soll. Dies vereinfacht die Weiterentwicklung vom Verhalten
aufgrund neuer Erkentnisse, da keine BenutzerInnenspezifische Einstellungen
berücksichtigt werden müssen. So können auch jeweils die optimalen Einstellungen
zur bestmöglichen Erreichung der Ziele (vgl. Kapitel
\ref{subsec:einleitun-motivation}) für alle gesetzt werden. Auch spezielle
Funktionen wie Updates sollen automatisch - also ohne BenutzerInneninteraktion
- stattfinden.

\subsubsection{Messverfahren}

So viele BenutzerInnen wie möglich werden die Applikation installiert. Nach der
Installation versuchen sie, irgendwie mit der Applikation zu interagieren. Dies
darf weder beim Start noch beim Beenden des Browsers möglich sein. Auch während
der gesamten Laufzeit des Browsers oder bei einem allfälligen speziellen
Funktion wie zum Beispiel einem Update darf die Erweiterung nicht wahrgenommen
werden. (Als Abgrenzung zum Verbrauchsverhalten (vgl. Kapitel 
\ref{subsec:nichtfunktionale-verbrauchsverhalten}), wo sie aufgrund des
Ressourcenverbrauchs nicht wahrgenommen werden darf, gilt hier einzig die
Wahrnehmung aufgrund von BenutzerInneninteraktionen wie zum Beispiel
Konfiguration oder Fehlermeldungen.)

\subsubsection{Geltungsbereich}

Diese Anforderung gilt für die gesamten Laufzeit der Applikation.
