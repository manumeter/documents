\subsection{Browserstart}
\label{subsec:funktionale-start}

Da die Applikation nicht direkt mit dem/der BenutzerIn interagiert, soll er/sie
auch nicht durch sie gestört werden (siehe dazu die nichtfunktionale Anforderung
\ref{subsec:nichtfunktionale-verbrauchsverhalten}). Es ist daher essenziell,
dass gerade beim Start der Applikation wenig Ressourcen (Rechenleistung und
Netzwerkverbindung) verbraucht werden. Dies bedingt, dass zu diesem Zeitpunkt
nur wenig gemacht wird.

Es wird also lediglich die Funktion\footnote[1]{Funktion meint in diesem
Zusammenhang lediglich die als funktionale Anforderung gruppierten
Ausführungsschritte und hat nichts mit dem Konstrukt der Funktion einer
Programmiersprache zu tun. Ob eine solche funktionale Anforderung später als
Funktion implementiert wird ist Sache der Software-Architektur.} gestartet,
welche dauerhaft im Hintergrund läuft, alles andere wird bei entsprechender
Ressourcenverfügbarkeit später (in ebendieser Funktion) erledigt.

\subsubsection{Auslöser}

Ausgelöst wird diese Funktion indirekt durch den/die BenutzerIn, welcheR den
Browser (mit \textit{Google Muddle} als Erweiterung) startet. Direkter Auslöser
ist somit der Browser.

\subsubsection{AkteurInnen}

Der/die BenutzerIn ist indirekt AkteurIn, da er/sie den Browser startet, welcher
wiederum \textit{Google Muddle} startet.

\subsubsection{Essenzielle Schritte}

\begin{enumerate}
\item Der/die BenutzerIn startet den Browser.
\item Der Browser initialisiert seine Erweiterungen, dabei wird auch
\textit{Google Muddle} gestartet.
\item Die \enquote{Eruierung der Ausführung} (funktionale Anforderung
\ref{subsec:funktionale-ausfuehrung}) wird gestartet.
\end{enumerate}
