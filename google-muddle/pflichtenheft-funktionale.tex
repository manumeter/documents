\section{Funktionale Anforderungen}
\label{sec:funktionale}

Folgende Tabelle (\ref{tbl:uebersicht-funktionale}) zeigt eine Übersicht aller
funktionalen Anforderungen an \textit{Google Muddle}.

Die funktionalen Anforderungen legen in ihrer Gesamtheit fest, welche Aufgaben
die Applikation zu erfüllen hat. Dazu werden neben den Auslösern auch
mögliche AkteurInnen und die essenziellen Schritte aufgeführt, aus welchen sich
die jeweilige Anforderung zusammensetzt. Die funktionalen Anforderungen gehen
dabei nicht auf die Eigenschaften ein, welche die Applikation neben der
Funktionalität auch noch zu erfüllen hat. Sie werden in Kapitel
\ref{sec:nichtfunktionale} definiert.

In der Grobspezifikation sollen aus diesen Anforderungen die Anwendungsfälle
abgeleitet werden können.
\\[\intextsep]
\begin{minipage}{\linewidth}
	\centering
	\begin{tabular}{rll}
		\hline
		Nummer                                & Bezeichnung               
		                                      & Quelle               \\
		\hline
		\ref{subsec:funktionale-start}        & Browserstart              
		                                      & Autor des Dokuments  \\
		\ref{subsec:funktionale-ausfuehrung}  & Eruieren der Ausführung   
		                                      & Autor des Dokuemnts  \\
		\ref{subsec:funktionale-suche}        & Suchanfrage               
		                                      & Autor des Dokuemnts  \\
		\ref{subsec:funktionale-daten}        & Datenrücksendung          
		                                      & Autor des Dokuemnts  \\
		\ref{subsec:funktionale-update}       & Update                    
		                                      & Autor des Dokuemnts  \\
		\ref{subsec:funktionale-end}          & Browserende               
		                                      & Autor des Dokuemnts  \\
		\hline
	\end{tabular}
	\tabcaption{Übersicht der funktionalen Anforderungen}
	\label{tbl:uebersicht-funktionale}
\end{minipage}

\newpage
\subsection{Browserstart}
\label{subsec:funktionale-start}

Da die Applikation nicht direkt mit dem/der BenutzerIn interagiert, soll er/sie
auch nicht durch sie gestört werden (siehe dazu die nichtfunktionale Anforderung
\ref{subsec:nichtfunktionale-verbrauchsverhalten}). Es ist daher essenziell,
dass gerade beim Start der Applikation wenig Ressourcen (Rechenleistung und
Netzwerkverbindung) verbraucht werden. Dies bedingt, dass zu diesem Zeitpunkt
nur wenig gemacht wird.

Es wird also lediglich die Funktion\footnote[1]{Funktion meint in diesem
Zusammenhang lediglich die als funktionale Anforderung gruppierten
Ausführungsschritte und hat nichts mit dem Konstrukt der Funktion einer
Programmiersprache zu tun. Ob eine solche funktionale Anforderung später als
Funktion implementiert wird ist Sache der Software-Architektur.} gestartet,
welche dauerhaft im Hintergrund läuft, alles andere wird bei entsprechender
Ressourcenverfügbarkeit später (in ebendieser Funktion) erledigt.

\subsubsection{Auslöser}

Ausgelöst wird diese Funktion indirekt durch den/die BenutzerIn, welcheR den
Browser (mit \textit{Google Muddle} als Erweiterung) startet. Direkter Auslöser
ist somit der Browser.

\subsubsection{AkteurInnen}

Der/die BenutzerIn ist indirekt AkteurIn, da er/sie den Browser startet, welcher
wiederum \textit{Google Muddle} startet.

\subsubsection{Essenzielle Schritte}

\begin{enumerate}
\item Der/die BenutzerIn startet den Browser.
\item Der Browser initialisiert seine Erweiterungen, dabei wird auch
\textit{Google Muddle} gestartet.
\item Die \enquote{Eruierung der Ausführung} (funktionale Anforderung
\ref{subsec:funktionale-ausfuehrung}) wird gestartet.
\end{enumerate}

\newpage
\subsection{Eruieren der Ausführung}
\label{subsec:funktionale-ausfuehrung}

Mit \enquote{Eruieren der Ausführung} ist gemeint, dass die Applikation ständig
überprüft, wie stark der Computer und die Netzwerkverbindung gerade ausgelastet
sind und, wenn genügend Ressourcen zur Verfügung stehen, Suchanfragen oder
Updates initiiert.

\subsubsection{Auslöser}

Auslöser ist der Start der Applikation (funktionale Anforderung
\ref{subsec:funktionale-start}).

\subsubsection{AkteurInnen}

AkteurInnen sind für diese funktionale Anforderung keine vorhanden.

\subsubsection{Essenzielle Schritte}

\begin{enumerate}
\item Die Applikation wartet eine zufällige Zeit zwischen 5 Sekunden und 2
Minuten\footnote[1]{Bei diesen Zeiten handelt es sich um Vorschläge, die im
Verlauf der Entwicklung durch Erfahrungswerte ersetzt werden sollten.}.
\item \textit{Google Muddle} überprüft, ob der Computer und die
Netzwerkverbindung ausgelastet sind. Falls eine Schwelle überschritten wird, die
für den/die BenutzerIn bemerkbar ist, wird Punkt 1 erneut ausgeführt. Um hier
eine Endlosschleife zu vermeiden, soll die Schwelle dynamisch an die
Gegebenheiten angepasst werden.
\item Falls die letzte Überprüfung schon über 7 Tage\footnote[2]{Die 7 Tage
werden als angemessen einschätzt. Auch hier darf die Zahl im Verlauf der
Entwicklung gerne angepasst werden, wenn die Erfahrung zeigt, dass ein anderer
Wert optimal wäre.} her ist überprüft \textit{Google Muddle}, ob eine neuere
Version verfügbar ist und installiert diese gegebenenfalls (funktionale
Anforderung \ref{subsec:funktionale-update}). Dabei wird der Zeitpunkt dieser
Überprüfung protokolliert. Sollte eine neuere Version installiert worden sein,
wird \textit{Google Muddle} neu gestartet.
\item Eine Suchanfrage wird gestartet (funktionale Anforderung
\ref{subsec:funktionale-suche}).
\item Die Schritte werden erneut ab Punkt 1 ausgeführt.
\end{enumerate}

\newpage
\subsection{Suchanfrage}
\label{subsec:funktionale-suche}

Soll eine neue Suchanfrage gestellt werden, wird zuerst auf die Datenbank mit
bisherigen Suchbegriffen aller Benutzer von \textit{Google Muddle} zugegriffen.
Dabei stellt der Server einen Suchbegriff zur Verfügung (siehe 
\ref{subsubsec:funktionale-suche-schritte} Essenzielle Schritte). Nachdem nach
diesem Begriff gesucht wurde, soll ein zufällig ausgewähltes Ergebnis geöffnet
werden. Dies lässt die Suche \enquote{echter} wirken und verfälscht weitere
Statistiken.

\subsubsection{Auslöser}

Auslöser ist das \enquote{Eruieren der Ausführung} (funktionale Anforderung
\ref{subsec:funktionale-ausfuehrung}) beziehungsweise die Zeit, da immer wieder
(in unterschiedlichen zufälligen Zeitinvervallen) Suchanfragen ausgelöst werden.

\subsubsection{AkteurInnen}

AkteurInnen sind für diese funktionale Anforderung keine vorhanden.

\subsubsection{Essenzielle Schritte}
\label{subsubsec:funktionale-suche-schritte}

\begin{enumerate}
\item Herunterladen eines Suchbegriffes. Der Suchbegriff wird vom Server auf
zufälliger Basis zur Verfügung gestellt, ohne, dass der Client (\textit{Google
Muddle}) darauf Einfluss nehmen könnte.\footnote[1]{Dies ermöglicht es, später
serverseitig neue Features einzubauen. So könnte zum Beispiel der reine Zufall
durch einen besseren Algorithmus ersetzt werden, der reale Trends abzuflachen
versucht oder vermehrt nach indizierten Wörtern sucht, um diesen
Überwachungsmechanismus zu stören.}
\item Bei Google nach dem erhaltenen Begriff (beziehungsweise dem Satz) suchen.
\item Ein von Google zur Verfügung gestelltes Suchresultat nach dem
Zufallsprinzip\footnote[2]{Auch dieser Zufallsmechanismus könnte später durch
einen \enquote{intelligenten} Algorithmus ersetzt werden.} öffnen.
\item Die Verbindung schliessen.
\end{enumerate}

\newpage
\subsection{Datenrücksendung}
\label{subsec:funktionale-daten}

Mit Datenrücksendung ist gemeint, dass echte Suchanfragen der BenutzerInnen an
eine Datenbank übermittelt werden, woraus sie dann für automatische Suchanfragen
verwendet werden können.

Diese funktionale Anforderung wird im Rahmen der Modularbeit nicht erfasst.

\subsection{Update}
\label{subsec:funktionale-update}

Der Update wird immer dann ausgeführt, wenn eine neuere Version von
\textit{Google Muddle} zur Verfügung steht.

Diese funktionale Anforderung wird im Rahmen der Modularbeit nicht erfasst.

\subsection{Browserende}
\label{subsec:funktionale-end}

Diese Funktion tritt in Kraft, wenn der Browser geschlossen wird.

Sie wird im Rahmen der Modularbeit nicht erfasst.

