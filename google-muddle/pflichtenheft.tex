% in PDF konfertieren mit (mehrfachdurchläufe notwendig) :! pdflatex %

\documentclass[a4paper, 11pt]{article}

\usepackage[utf8]{inputenc} % für Umlaut-Encoding
\usepackage{ngerman} % für deutsche Titel, Silbentrennung, etc. (benötigt das
                     % Paket texlive-lang-german)
\usepackage{graphicx} % für Bildeinbindung
\usepackage[round]{natbib} % für das Literaturverzeichnis
\usepackage{acronym} % für das Abkürzungsverzeichnis (benötigt das Paket
                     % texlive-latex-extra)
\usepackage{multirow} % für Tabellen mit spaltenübergreifenden Werten
\usepackage{nonfloat} % für Tabellen/Bild-Positionierung
\usepackage[autostyle=true,german=quotes]{csquotes} % Anführungszeichen global
                                                    % definieren
\usepackage[hidelinks]{hyperref} % für URLs

\title{Pflichtenheft 0.8 \textit{Google Muddle}}
\author{Manuel Mästinger\\\small Modularbeit 4.2 Software Engineering\\\small
ZHAW, School of Engineering, MAS Informatik 8}

\begin{document}

\maketitle
\newpage

\section*{Änderungsverlauf}

Tabelle \ref{tbl:aenderungen} zeigt Änderungen, die im Verlauf der Bearbeitung
an diesem Dokument vorgenommen wurden.
\\[\intextsep]
\begin{minipage}{\linewidth}
	\centering
	\begin{tabular}{rlll}
		\hline
		Version & Beschreibung                                  & Datum       \\
		\hline
		0.1     & Einleitung geschrieben                        & 25.03.2015  \\
		0.2     & Literatur und Abkürzungsverzeichnis eingefügt & 25.03.2015  \\
		0.3     & Grobstrukturierung des Dokuments              & 27.03.2015  \\
		0.4     & Mit nichtfunktionalen Anforderungen begonnen  & 17.04.2015  \\
		0.5     & Nichtfunktionale Anforderungen erweitert      & 10.05.2015  \\
		0.6     & Funktionalen Anforderungen                    & 16.05.2015  \\
		0.7     & Überarbeitung des gesamten Dokuments          & 17.05.2015  \\
		0.8     & Wenige Korrekturen und Ergänzungen            & 14.06.2015  \\
		\hline
	\end{tabular}
	\tabcaption{Änderungen am Dokument}
	\label{tbl:aenderungen}
\end{minipage}


\newpage

\tableofcontents
\newpage

\section{Einleitung}
\label{sec:einleitung}

\subsection{Zum Dokument}

Dieses Dokument soll als möglichst präzise Vorlage für die
Grobspezifikation der Applikation \textit{Google Muddle} dienen.

Neben den funktionalen (Kapitel \ref{sec:funktionale}) und den nichtfunktionalen
Anforderungen (Kapitel \ref{sec:nichtfunktionale}) werden in diesem Dokument
auch die Rahmenbedingungen (Kapitel \ref{sec:rahmenbedingungen}) festgelegt.
Ergänzend sind im Kapitel \ref{sec:einleitung} die Motivation sowie der
angestrebte Soll-Zustand kurz skizziert.

Am Ende des Dokuments können dann die Verweise auf zitierte Literatur, das
Abkürzungens- und das Tabellenverzeichnis gefunden werden.

Um die Geschlechtsneutralität der Aussagen zu gewährleisten werden in diesem
Dokument in der Regel Binnenmajuskeln verwendet. So sollen die weibliche und
die männliche Bezeichnung in kurzer Form vereint werden (vgl.
\url{https://de.wikipedia.org/wiki/Binnen-I}).

\subsection{Referenzdokumente}

Es sind keine Referenzdokumente vorhanden.

\subsection{Motivation}
\label{subsec:einleitun-motivation}

Wie die Enthüllungen durch \cite{snowden:nsa} gezeigt haben, werden (durch die
\acs{NSA}) flächendeckend Überwachungstechniken eingesetzt, die das Verhalten
fast aller Internetbesucher protokollieren und automatisch auswerten.
Multinationale Internetkonzerne wie Google mit strategisch gut geeignetem
Dienstleistungsangebot und ausreichendem Kundenstamm spielen dabei eine zentrale
Rolle. Um dieser golbalen Überwachung zu entgehen, oder sie wenigstens zu
erschweren, gibt es meines Erachtens drei Strategien, deren Kombination
höchstmöglichen Schutz der Privatsphäre bietet.

Erstens kann versucht werden, Nutz- und Metadaten so zu verschlüsseln bzw. zu
verschleiern (zum Beispiel durch die Umleitung über viele kaskadierte Knoten),
dass deren Protokollierung nutzlos würde, da die Auswertung kaum mehr möglich
wäre. Da diese Strategie sehr aufwendig ist und dazu noch von einer grossen
Anzahl BenutzerInnen umgesetzt werden müsste, damit die jeweilige Kommunikation
geschützt wäre, ist ihre baldige Verbreitung trotz vorhandener Werkzeuge meiner
Einschätzung nach eher unwahrscheinlich.

Zweitens könnte der Versuch unternommen werden, die überwachenden Unternehmen
und die überwachten Internetknoten komplett zu meiden. Dies würde faktisch zum
totalen Verzicht auf das Internet führen, da praktisch der gesamte Datenverkehr
über wenige einzelne (überwachte) Knoten verläuft. Zudem werden die am stärksten
genutzten Dienstleistung von einigen wenigen Konzernen angeboten, die gemäss
\cite{snowden:nsa} oft ebenfalls zur (geheimen) Überwachtung verpflichtet sind.

Die dritte Strategie, auf welche ich in diesem Projekt zurückgreifen werde,
funktioniert komplett anders. Anstatt die Nutz- und Metadaten so zu
manipulieren, dass sie nicht mehr ausgewertet werden können oder auf grosse
Teile des Internets zu verzichten, wird mit dieser Strategie versucht, die
echten Daten durch grosse Mengen von künstlich erzeugten Daten zu erweitern, die
sich kaum von den Echten unterscheiden lassen. Durch die entstehende Datenflut
wird die Auswertung der Protokolle stark verfälscht und nutzlos.

\subsection{Soll-Zustand}

Um diese dritte Strategie umzusetzen, sind verschiedene Werkzeuge, die jeweils
auf das verwendete Kommunikationsmittel bzw. die Dienstleistung angepasst sind,
notwendig. Um einen grossen Bereich der Internetnutzung abzudecken, soll in
diesem Projekt eine Erweiterung für Internet-Browser geschrieben werden, welche
die Google Suche wie folgt nutzt.

In der gesamten Zeit, in der ein Browser mit dieser Erweiterung läuft, sollen im
Hintergrund (ohne, dass der/die BenutzerIn dadurch gestört wird) Suchanfragen an
Google gestellt werden. Die Suchanfragen sowie die zeitlichen Abstände zwischen
einzelnen Anfragen sollen dabei möglichst nicht von echten Suchanfragen
unterschieden werden können. Anschliessend soll ein zufällig ausgewähltes
Suchresultat geöffnet werden. Als Quelle für Suchbegriffe könnten zum Beispiel
echte Suchanfragen anderer BenutzerInnen der Browser-Erweiterung dienen.

Diese Methode hat ausserdem den positiven Nebeneffekt, die so genannte
Filterblase zu verhindern. Unter Filterblase wird der Mechanismus verstanden,
dass Suchmaschinen BenutzerInnenprofile anlegen und den jeweiligen BenutzerInnen
auf sie zugeschnittene Ergebnisse anzeigen. Dabei entgehen den Leuten wichtige
Alternativen, die vom Algorithmus bereis ausgefiltert wurden.

Dabei soll sich diese Arbeit ausschliesslich mit dem Client-Teil der Software
beschäftigen. Allfällige Server-Komponenten sollen vorerst nicht berücksichtigt
werden.

\newpage
\section{Funktionale Anforderungen}
\label{sec:funktionale}

Folgende Tabelle (\ref{tbl:uebersicht-funktionale}) zeigt eine Übersicht aller
funktionalen Anforderungen an \textit{Google Muddle}.

Die funktionalen Anforderungen legen in ihrer Gesamtheit fest, welche Aufgaben
die Applikation zu erfüllen hat. Dazu werden neben den Auslösern auch
mögliche AkteurInnen und die essenziellen Schritte aufgeführt, aus welchen sich
die jeweilige Anforderung zusammensetzt. Die funktionalen Anforderungen gehen
dabei nicht auf die Eigenschaften ein, welche die Applikation neben der
Funktionalität auch noch zu erfüllen hat. Sie werden in Kapitel
\ref{sec:nichtfunktionale} definiert.

In der Grobspezifikation sollen aus diesen Anforderungen die Anwendungsfälle
abgeleitet werden können.
\\[\intextsep]
\begin{minipage}{\linewidth}
	\centering
	\begin{tabular}{rll}
		\hline
		Nummer                                & Bezeichnung               
		                                      & Quelle               \\
		\hline
		\ref{subsec:funktionale-start}        & Browserstart              
		                                      & Autor des Dokuments  \\
		\ref{subsec:funktionale-ausfuehrung}  & Eruieren der Ausführung   
		                                      & Autor des Dokuemnts  \\
		\ref{subsec:funktionale-suche}        & Suchanfrage               
		                                      & Autor des Dokuemnts  \\
		\ref{subsec:funktionale-daten}        & Datenrücksendung          
		                                      & Autor des Dokuemnts  \\
		\ref{subsec:funktionale-update}       & Update                    
		                                      & Autor des Dokuemnts  \\
		\ref{subsec:funktionale-end}          & Browserende               
		                                      & Autor des Dokuemnts  \\
		\hline
	\end{tabular}
	\tabcaption{Übersicht der funktionalen Anforderungen}
	\label{tbl:uebersicht-funktionale}
\end{minipage}

\newpage
\subsection{Browserstart}
\label{subsec:funktionale-start}

Da die Applikation nicht direkt mit dem/der BenutzerIn interagiert, soll er/sie
auch nicht durch sie gestört werden (siehe dazu die nichtfunktionale Anforderung
\ref{subsec:nichtfunktionale-verbrauchsverhalten}). Es ist daher essenziell,
dass gerade beim Start der Applikation wenig Ressourcen (Rechenleistung und
Netzwerkverbindung) verbraucht werden. Dies bedingt, dass zu diesem Zeitpunkt
nur wenig gemacht wird.

Es wird also lediglich die Funktion\footnote[1]{Funktion meint in diesem
Zusammenhang lediglich die als funktionale Anforderung gruppierten
Ausführungsschritte und hat nichts mit dem Konstrukt der Funktion einer
Programmiersprache zu tun. Ob eine solche funktionale Anforderung später als
Funktion implementiert wird ist Sache der Software-Architektur.} gestartet,
welche dauerhaft im Hintergrund läuft, alles andere wird bei entsprechender
Ressourcenverfügbarkeit später (in ebendieser Funktion) erledigt.

\subsubsection{Auslöser}

Ausgelöst wird diese Funktion indirekt durch den/die BenutzerIn, welcheR den
Browser (mit \textit{Google Muddle} als Erweiterung) startet. Direkter Auslöser
ist somit der Browser.

\subsubsection{AkteurInnen}

Der/die BenutzerIn ist indirekt AkteurIn, da er/sie den Browser startet, welcher
wiederum \textit{Google Muddle} startet.

\subsubsection{Essenzielle Schritte}

\begin{enumerate}
\item Der/die BenutzerIn startet den Browser.
\item Der Browser initialisiert seine Erweiterungen, dabei wird auch
\textit{Google Muddle} gestartet.
\item Die \enquote{Eruierung der Ausführung} (funktionale Anforderung
\ref{subsec:funktionale-ausfuehrung}) wird gestartet.
\end{enumerate}

\newpage
\subsection{Eruieren der Ausführung}
\label{subsec:funktionale-ausfuehrung}

Mit \enquote{Eruieren der Ausführung} ist gemeint, dass die Applikation ständig
überprüft, wie stark der Computer und die Netzwerkverbindung gerade ausgelastet
sind und, wenn genügend Ressourcen zur Verfügung stehen, Suchanfragen oder
Updates initiiert.

\subsubsection{Auslöser}

Auslöser ist der Start der Applikation (funktionale Anforderung
\ref{subsec:funktionale-start}).

\subsubsection{AkteurInnen}

AkteurInnen sind für diese funktionale Anforderung keine vorhanden.

\subsubsection{Essenzielle Schritte}

\begin{enumerate}
\item Die Applikation wartet eine zufällige Zeit zwischen 5 Sekunden und 2
Minuten\footnote[1]{Bei diesen Zeiten handelt es sich um Vorschläge, die im
Verlauf der Entwicklung durch Erfahrungswerte ersetzt werden sollten.}.
\item \textit{Google Muddle} überprüft, ob der Computer und die
Netzwerkverbindung ausgelastet sind. Falls eine Schwelle überschritten wird, die
für den/die BenutzerIn bemerkbar ist, wird Punkt 1 erneut ausgeführt. Um hier
eine Endlosschleife zu vermeiden, soll die Schwelle dynamisch an die
Gegebenheiten angepasst werden.
\item Falls die letzte Überprüfung schon über 7 Tage\footnote[2]{Die 7 Tage
werden als angemessen einschätzt. Auch hier darf die Zahl im Verlauf der
Entwicklung gerne angepasst werden, wenn die Erfahrung zeigt, dass ein anderer
Wert optimal wäre.} her ist überprüft \textit{Google Muddle}, ob eine neuere
Version verfügbar ist und installiert diese gegebenenfalls (funktionale
Anforderung \ref{subsec:funktionale-update}). Dabei wird der Zeitpunkt dieser
Überprüfung protokolliert. Sollte eine neuere Version installiert worden sein,
wird \textit{Google Muddle} neu gestartet.
\item Eine Suchanfrage wird gestartet (funktionale Anforderung
\ref{subsec:funktionale-suche}).
\item Die Schritte werden erneut ab Punkt 1 ausgeführt.
\end{enumerate}

\newpage
\subsection{Suchanfrage}
\label{subsec:funktionale-suche}

Soll eine neue Suchanfrage gestellt werden, wird zuerst auf die Datenbank mit
bisherigen Suchbegriffen aller Benutzer von \textit{Google Muddle} zugegriffen.
Dabei stellt der Server einen Suchbegriff zur Verfügung (siehe 
\ref{subsubsec:funktionale-suche-schritte} Essenzielle Schritte). Nachdem nach
diesem Begriff gesucht wurde, soll ein zufällig ausgewähltes Ergebnis geöffnet
werden. Dies lässt die Suche \enquote{echter} wirken und verfälscht weitere
Statistiken.

\subsubsection{Auslöser}

Auslöser ist das \enquote{Eruieren der Ausführung} (funktionale Anforderung
\ref{subsec:funktionale-ausfuehrung}) beziehungsweise die Zeit, da immer wieder
(in unterschiedlichen zufälligen Zeitinvervallen) Suchanfragen ausgelöst werden.

\subsubsection{AkteurInnen}

AkteurInnen sind für diese funktionale Anforderung keine vorhanden.

\subsubsection{Essenzielle Schritte}
\label{subsubsec:funktionale-suche-schritte}

\begin{enumerate}
\item Herunterladen eines Suchbegriffes. Der Suchbegriff wird vom Server auf
zufälliger Basis zur Verfügung gestellt, ohne, dass der Client (\textit{Google
Muddle}) darauf Einfluss nehmen könnte.\footnote[1]{Dies ermöglicht es, später
serverseitig neue Features einzubauen. So könnte zum Beispiel der reine Zufall
durch einen besseren Algorithmus ersetzt werden, der reale Trends abzuflachen
versucht oder vermehrt nach indizierten Wörtern sucht, um diesen
Überwachungsmechanismus zu stören.}
\item Bei Google nach dem erhaltenen Begriff (beziehungsweise dem Satz) suchen.
\item Ein von Google zur Verfügung gestelltes Suchresultat nach dem
Zufallsprinzip\footnote[2]{Auch dieser Zufallsmechanismus könnte später durch
einen \enquote{intelligenten} Algorithmus ersetzt werden.} öffnen.
\item Die Verbindung schliessen.
\end{enumerate}

\newpage
\subsection{Datenrücksendung}
\label{subsec:funktionale-daten}

Mit Datenrücksendung ist gemeint, dass echte Suchanfragen der BenutzerInnen an
eine Datenbank übermittelt werden, woraus sie dann für automatische Suchanfragen
verwendet werden können.

Diese funktionale Anforderung wird im Rahmen der Modularbeit nicht erfasst.

\subsection{Update}
\label{subsec:funktionale-update}

Der Update wird immer dann ausgeführt, wenn eine neuere Version von
\textit{Google Muddle} zur Verfügung steht.

Diese funktionale Anforderung wird im Rahmen der Modularbeit nicht erfasst.

\subsection{Browserende}
\label{subsec:funktionale-end}

Diese Funktion tritt in Kraft, wenn der Browser geschlossen wird.

Sie wird im Rahmen der Modularbeit nicht erfasst.


\newpage
\section{Nichtfunktionale Anforderungen}
\label{sec:nichtfunktionale}

Eine Anforderung ist gemäss \cite{glinz:nfr} als nichtfunktional zu verstehen, 
wenn \enquote{das ihr zu Grunde liegende Bedürfnis eine nicht gegenständliche
Eigenschaft [der Applikation] ist}. In diesem Kapitel werden also Anforderungen
definiert, bei welchen das (der Anforderung zu Grunde liegende) Bedürfnis nicht
die Funktionalität selbst ist.

In Tabelle \ref{tbl:softwarequalitaet} ist die (vollständige) Liste der
Qualitätsmerkmale von Software gemäss \acs{ISO}/\acs{IEC} 9126 (\acs{DIN} 66272)
zu finden. Einige davon werden im Folgenden als nichtfunktionale Anforderungen
in Bezug auf \textit{Google Muddle} genauer beschrieben. Die übrigen werden im
Rahmen dieser Modularbeit nicht erfasst.
\\[\intextsep]
\begin{minipage}{\linewidth}
	\centering
	\begin{tabular}{llr}
		\hline
		Kategorie & Qualitätsmerkmal & Priorität \\
		\hline
		\multirow{6}{*}{Funktionalität:} & Angemessenheit & mittel \\
		 & \textbf{Richtigkeit} & \textbf{hoch} \\
		 & \textbf{Interoperabilität}
		  (\ref{subsec:nichtfunktionale-interoperabilitaet}) & \textbf{hoch} \\
		 & \textbf{Sicherheit} & \textbf{hoch} \\
		 & Ordnungsmässigkeit & mittel \\
		 & Konformität & gering \\
		\hline
		\multirow{4}{*}{Zuverlässigkeit:} & Reife & mittel \\
		 & \textbf{Fehlertoleranz} & \textbf{hoch} \\
		 & Wiederherstellbarkeit & gering \\
		 & Konformität & gering \\
		\hline
		\multirow{5}{*}{Benutzbarkeit:} & Verständlichkeit & mittel \\
		 & Erlernbarkeit & gering \\
		 & \textbf{Bedienbarkeit}
		  (\ref{subsec:nichtfunktionale-bedienbarkeit}) & \textbf{hoch} \\
		 & \textbf{Attraktivität} & \textbf{hoch} \\
		 & Konformität & gering \\
		\hline
		\multirow{3}{*}{Effizienz:} & Zeitverhalten & mittel \\
		 & \textbf{Verbrauchsverhalten}
		  (\ref{subsec:nichtfunktionale-verbrauchsverhalten}) & \textbf{hoch} \\
		 & Konformität & gering \\
		\hline
		\multirow{5}{*}{Wartbarkeit/Änderbarkeit:} & \textbf{Analysierbarkeit}
		 & \textbf{hoch} \\
		 & \textbf{Modifizierbarkeit} & \textbf{hoch} \\
		 & \textbf{Stabilität} & \textbf{hoch} \\
		 & Testbarkeit & mittel \\
		 & Konformität & gering \\
		\hline
		\multirow{5}{*}{Übertragbarkeit:} & \textbf{Anpassbarkeit}
		 & \textbf{hoch} \\
		 & \textbf{Installierbarkeit}
		  (\ref{subsec:nichtfunktionale-installierbarkeit}) & \textbf{hoch} \\
		 & Koexistenz & gering \\
		 & Austauschbarkeit & gering \\
		 & Konformität & gering \\
		\hline
	\end{tabular}
	\tabcaption{Qualitätsmerkmale nach \cite{iso:9126}}
	\label{tbl:softwarequalitaet}
\end{minipage}
\newpage

\subsection{Interoperabilität}
\label{subsec:nichtfunktionale-interoperabilitaet}

Interoperabilität bezeichnet gemäss \cite{iso:9126} die \enquote{Fähigkeit, mit
vorgegebenen Systemen zusammenzuwirken}.

\subsubsection{Beschreibung}

Es wird ein hoher Verbreitungsgrad der Applikation angestrebt, da sich dadurch
ihre Wirkung verbessert. Deshalb ist es wichtig, eine möglichst breite Variation
von Browsern zu unterstützen. Die Applikation soll mit den in der Tabelle
\ref{tbl:browser-gewichtung} aufgelisteten Browsern (gemäss Gewichtung) in
jeweils neuester Version (für Desktop-Betriebssysteme) kompatibel sein. Die
Gewichtung wurde auf Grundlage der Browserstatistik von
\cite{w3schools:browserstats} vorgenommen.
\\[\intextsep]
\begin{minipage}{\linewidth}
	\centering
	\begin{tabular}{llr}
		\hline
		Browser           & Betriebssysteme     & Gewichtung $g$ \\
		\hline
		Apple Safari      & Mac                 & \textbf{1} \\
		Google Chrome     & Linux, Mac, Windows & \textbf{4} \\
		Internet Explorer & Windows             & \textbf{2} \\
		Mozilla Firefox   & Linux, Mac, Windows & \textbf{3} \\
		Opera             & Mac, Windows        & \textbf{1} \\
		\hline
	\end{tabular}
	\tabcaption{Liste kompatibler Browser mit Gewichtung}
	\label{tbl:browser-gewichtung}
\end{minipage}

\subsubsection{Messverfahren}

Zwecks Messbarkeit wurden drei Stufen der Erfüllung, die pro Browser ermittelt
werden, definiert (siehe Tabelle \ref{tbl:interoperabilitaet-erfuellung}).
\\[\intextsep]
\begin{minipage}{\linewidth}
	\centering
	\begin{tabular}{ll}
		\hline
		Erfüllungsgrad $e$ & Beschreibung \\
		\hline
		\textbf{1}
		 & Die Applikation läuft mit diesem Browser. \\
		\multirow{2}{*}{\textbf{2}}
		 & Die Applikation läuft mit diesem Browser und \\
		 & bietet einen reduzierten Funktionsumfang. \\
		\multirow{2}{*}{\textbf{3}}
		 & Die Applikation läuft mit diesem Browser und \\
		 & bietet den vollen Funktionsumfang. \\
		\hline
	\end{tabular}
	\tabcaption{Erfüllungsgrad der Interoperabilität pro Browser}
	\label{tbl:interoperabilitaet-erfuellung}
\end{minipage}
\\[\intextsep]
Damit die Anforderung als erfüllt betrachtet wird, ist eine Summe für alle
Browser aus Tabelle \ref{tbl:browser-gewichtung} aus Gewichtung $g$
multipliziert mit Erfüllungsgrad $e$ (also $ \sum g * e $) von mindestens 29
erforderlich. Maximal ist 33 möglich.

\subsubsection{Geltungsbereich}

Diese Anforderung bezieht sich auf alle funktionalen Anforderungen, falls der
Erfüllungsgrad 3 ist. Ansonsten (Erfüllungsgrad 1 und 2) kann die Untermenge
beliebig ausgewählt werden (siehe Tabelle
\ref{tbl:interoperabilitaet-erfuellung}).

\newpage
\subsection{Bedienbarkeit}
\label{subsec:nichtfunktionale-bedienbarkeit}

Gemeint ist gemäss \cite{iso:9126} der \enquote{Aufwand für [...]
Benutzer[Innen], die Anwendung zu bedienen}.

\subsubsection{Beschreibung}

Da \textit{Google Muddle} vollständig automatisch funktionieren muss, ist keine
Interatktion mit BenutzerInnen gewünscht. Sie sollen auch nicht die Möglichkeit
haben, Konfigurationen vorzunehmen, da die Verhaltensweise der Applikation für
jedeN identisch sein soll. Dies vereinfacht die Weiterentwicklung vom Verhalten
aufgrund neuer Erkentnisse, da keine BenutzerInnenspezifische Einstellungen
berücksichtigt werden müssen. So können auch jeweils die optimalen Einstellungen
zur bestmöglichen Erreichung der Ziele (vgl. Kapitel
\ref{subsec:einleitun-motivation}) für alle gesetzt werden. Auch spezielle
Funktionen wie Updates sollen automatisch - also ohne BenutzerInneninteraktion
- stattfinden.

\subsubsection{Messverfahren}

So viele BenutzerInnen wie möglich werden die Applikation installiert. Nach der
Installation versuchen sie, irgendwie mit der Applikation zu interagieren. Dies
darf weder beim Start noch beim Beenden des Browsers möglich sein. Auch während
der gesamten Laufzeit des Browsers oder bei einem allfälligen speziellen
Funktion wie zum Beispiel einem Update darf die Erweiterung nicht wahrgenommen
werden. (Als Abgrenzung zum Verbrauchsverhalten (vgl. Kapitel 
\ref{subsec:nichtfunktionale-verbrauchsverhalten}), wo sie aufgrund des
Ressourcenverbrauchs nicht wahrgenommen werden darf, gilt hier einzig die
Wahrnehmung aufgrund von BenutzerInneninteraktionen wie zum Beispiel
Konfiguration oder Fehlermeldungen.)

\subsubsection{Geltungsbereich}

Diese Anforderung gilt für die gesamten Laufzeit der Applikation.

\newpage
\subsection{Verbrauchsverhalten}
\label{subsec:nichtfunktionale-verbrauchsverhalten}

Verbrauchsverhalten definiert nach \cite{iso:9126} die \enquote{Anzahl und Dauer
der benötigten Betriebsmittel bei der Erfüllung der Funktionen}.

\subsubsection{Beschreibung}

Da \textit{Google Muddle} für seine BenutzerInnen direkt funktional wenig
bietet, soll die Applikation auch möglichst unaufdringlich und
ressourcenschonend arbeiten. Es ist nicht sehr relevant, wann die einzelnen
Funktionen genau ausgeführt werden. Dies sollte daher möglichst dann geschehen,
wenn der Computer wenig zu tun hat - also etwa wenn niemand damit am arbeiten
ist.

\subsubsection{Messverfahren}

So viele BenutzerInnen wie möglich werden in einer Testphase mit einem Browser
arbeiten, ohne zu wissen, ob die Applikation installiert ist oder nicht. Dabei
dürfen sie keinen Unterschied feststellen. Die BenutzerInnen werden sowohl mit
aktueller, als auch mit alter Hardware arbeiten. Auch die Betriebssysteme und
die darauf installierten Browser werden divers sein (siehe Tabelle 
\ref{tbl:browser-gewichtung} auf Seite \pageref{tbl:browser-gewichtung}).

Wenn mehr als 5 Prozent der BenutzerInnen eindeutig erkennen, ob die Applikation
installiert ist oder nicht, gilt diese Anforderung als nicht erfüllt.

\subsubsection{Geltungsbereich}

Diese Anforderung gilt für die gesamten Laufzeit der Applikation. Als besonders
kritisch gilt dabei meist der Start, da in dieser Phase bewusst gewartet wird.

\newpage
\subsection{Installierbarkeit}
\label{subsec:nichtfunktionale-installierbarkeit}

\cite{iso:9126} definiert die Installierbarkeit als \enquote{Aufwand, der zum
Installieren der Software in einer festgelegten Umgebung notwendig ist}.

\subsubsection{Beschreibung}

Die Applikation soll im für den jweiligen Browser offiziellen Store für
Erweiterungen verfügbar sein.

Ist dies aufgrund von Einschränkungen oder Ausschluss seitens der BetreiberInnen
dieses Stores nicht möglich, soll die Applikation über andere Plattformen zum
herunterladen angeboten werden. Falls auf diese Möglichkeit zurückgegriffen
werden muss, soll auf der Webseite dieser Plattform zusätzlich zur Applikation
eine einfache Installationsanleitung präsentiert werden.

\subsubsection{Messverfahren}

\textit{Google Muddle} muss - in jedem der Browser aus Tabelle
\ref{tbl:browser-gewichtung} auf Seite \pageref{tbl:browser-gewichtung} - über
eine via Menü erreichbare Suchfunktion für Erweiterungen gefunden werden können.
Anschliessend soll sie, über den für den jeweiligen Browser üblichen Weg,
installiert werden können.

Ist die Applikation installiert, soll sie, ebenfalls über den für den jeweiligen
Browser üblichen Weg, vollständig deinstalliert werden können.

Wird auf die oben erwähnte Ausweichmöglichkeit einer anderen Plattform
zurückgegriffen, wird durch Rückfragen bei den BetreiberInnen geprüft, ob dies
wirklich aufgrund von Einschränkungen oder Ausschluss seitens der BetreiberInnen
geschehen ist. Sollte das nicht der Fall (gewesen) sein, gilt diese Anforderung
als nicht erfüllt.

\subsubsection{Geltungsbereich}

Die Installierbarkeit hat keine Beziehung zu irgend einer funktionalen
Anforderung.



\newpage
\section{Rahmenbedingungen}
\label{sec:rahmenbedingungen}

Die weiteren Bedingungen werden im Rahmen der Modularbeit nicht erfasst.


\begin{thebibliography}{Literaturverzeichnis}
	
	\bibitem[Glinz(2006)]{glinz:nfr} Glinz, Martin. (2006). Requirements
	Engineering I - Nicht-funktionale Anforderungen. Heruntergeladen von
	\url{https://files.ifi.uzh.ch/rerg/arvo/ftp/re\_I/Kapitel\_11\_NFAnf.pdf} am
	27.03.2014
	
	\bibitem[ISO/IEC 9126-1(2001)]{iso:9126} \acs{ISO}/\acs{IEC} 9126-1 (2001).
	Software engineering - Product quality - Part 1: Quality model.
	International Organization for Standardization.
	
	\bibitem[Snowden(2013)]{snowden:nsa} Snowden, Edward. (2013). \acs{NSA}
	interne Dokumente. Heruntergeladen von
	\url{https://edwardsnowden.com/2013/10/14/miscellaneous-leaked-documents/}
	am 25.03.2014
	
	\bibitem[W3Schools(2015)]{w3schools:browserstats} W3Schools (2015). Browser
	Statistics. Heruntergeladen von
	\url{http://www.w3schools.com/browsers/browsers_stats.asp} am 17.04.2014

\end{thebibliography}

\section*{Abkürzungen}
\begin{acronym}[TDMA]
	\setlength{\itemsep}{-\parsep}
	\acro{DIN}{Deutsches Institut für Normung}
	\acro{IEC}{International Electrotechnical Commission, ein Normungsgremium
	für Elektrotechnik}
	\acro{ISO}{International Organization for Standardization}
	\acro{NSA}{National Security Agency der \acs{USA}}
	\acro{USA}{United States of America}
\end{acronym}

\listoftables

%\listoffigures


\end{document}
