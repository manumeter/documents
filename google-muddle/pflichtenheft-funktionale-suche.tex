\subsection{Suchanfrage}
\label{subsec:funktionale-suche}

Soll eine neue Suchanfrage gestellt werden, wird zuerst auf die Datenbank mit
bisherigen Suchbegriffen aller Benutzer von \textit{Google Muddle} zugegriffen.
Dabei stellt der Server einen Suchbegriff zur Verfügung (siehe 
\ref{subsubsec:funktionale-suche-schritte} Essenzielle Schritte). Nachdem nach
diesem Begriff gesucht wurde, soll ein zufällig ausgewähltes Ergebnis geöffnet
werden. Dies lässt die Suche \enquote{echter} wirken und verfälscht weitere
Statistiken.

\subsubsection{Auslöser}

Auslöser ist das \enquote{Eruieren der Ausführung} (funktionale Anforderung
\ref{subsec:funktionale-ausfuehrung}) beziehungsweise die Zeit, da immer wieder
(in unterschiedlichen zufälligen Zeitinvervallen) Suchanfragen ausgelöst werden.

\subsubsection{AkteurInnen}

AkteurInnen sind für diese funktionale Anforderung keine vorhanden.

\subsubsection{Essenzielle Schritte}
\label{subsubsec:funktionale-suche-schritte}

\begin{enumerate}
\item Herunterladen eines Suchbegriffes. Der Suchbegriff wird vom Server auf
zufälliger Basis zur Verfügung gestellt, ohne, dass der Client (\textit{Google
Muddle}) darauf Einfluss nehmen könnte.\footnote[1]{Dies ermöglicht es, später
serverseitig neue Features einzubauen. So könnte zum Beispiel der reine Zufall
durch einen besseren Algorithmus ersetzt werden, der reale Trends abzuflachen
versucht oder vermehrt nach indizierten Wörtern sucht, um diesen
Überwachungsmechanismus zu stören.}
\item Bei Google nach dem erhaltenen Begriff (beziehungsweise dem Satz) suchen.
\item Ein von Google zur Verfügung gestelltes Suchresultat nach dem
Zufallsprinzip\footnote[2]{Auch dieser Zufallsmechanismus könnte später durch
einen \enquote{intelligenten} Algorithmus ersetzt werden.} öffnen.
\item Die Verbindung schliessen.
\end{enumerate}
