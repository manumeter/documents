\section{Einleitung}
\label{sec:einleitung}

\subsection{Zum Dokument}

Dieses Dokument soll als möglichst präzise und vollständige Vorlage für die
Feinspezifikation der Applikation \textit{Google Muddle} dienen.

In Kapitel \ref{sec:systemabgrenzung} wird das System \textit{Google Muddle}
gegen aussen abgegrenzt. Basierend auf den funktionalen Anforderungen aus dem
Pflichtenheft werden dann in diesem Dokument Anwendungsfälle (Kapitel
\ref{sec:anwendungsfaelle}) definiert. Anschliessend werden alle Schnittstellen
des Systems (Kapitel \ref{sec:systemschnittstellen}) spezifiziert. In Kapitel 
\ref{sec:ablaufbeschreibung} wird dann der Ablauf jedes Anwendungsfalles in
einem Diagramm dargestellt und beschrieben. Und in Kapitel
\ref{sec:klassendiagramm} ist das Klassendiagramm für das gesamte System zu
finden.

Am Ende des Dokuments können dann das Abkürzungens-, das Tabellen- und das
Abbildungsverzeichnis gefunden werden.

Um die Geschlechtsneutralität der Aussagen zu gewährleisten werden in diesem
Dokument in der Regel Binnenmajuskeln verwendet. So sollen die weibliche und die
männliche Bezeichnung in kurzer Form vereint werden (vgl.
\url{https://de.wikipedia.org/wiki/Binnen-I}).

Die aufgezeigten Diagramme verwenden alle, sofern nicht im Einzelnen anders
definiert, Modelle der \acs{UML} und bei Bedarf Prosatext als Ergänzung.

\subsection{Referenzdokumente}

Dieses Dokument baut auf dem Pflichtenheft für \textit{Google Muddle} auf. Vor
allem der Abschnitt Einleitung sollte zwecks Verständnis der Problemstellung
gelesen werden.
