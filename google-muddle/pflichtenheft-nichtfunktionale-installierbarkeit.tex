\subsection{Installierbarkeit}
\label{subsec:nichtfunktionale-installierbarkeit}

\cite{iso:9126} definiert die Installierbarkeit als \enquote{Aufwand, der zum
Installieren der Software in einer festgelegten Umgebung notwendig ist}.

\subsubsection{Beschreibung}

Die Applikation soll im für den jweiligen Browser offiziellen Store für
Erweiterungen verfügbar sein.

Ist dies aufgrund von Einschränkungen oder Ausschluss seitens der BetreiberInnen
dieses Stores nicht möglich, soll die Applikation über andere Plattformen zum
herunterladen angeboten werden. Falls auf diese Möglichkeit zurückgegriffen
werden muss, soll auf der Webseite dieser Plattform zusätzlich zur Applikation
eine einfache Installationsanleitung präsentiert werden.

\subsubsection{Messverfahren}

\textit{Google Muddle} muss - in jedem der Browser aus Tabelle
\ref{tbl:browser-gewichtung} auf Seite \pageref{tbl:browser-gewichtung} - über
eine via Menü erreichbare Suchfunktion für Erweiterungen gefunden werden können.
Anschliessend soll sie, über den für den jeweiligen Browser üblichen Weg,
installiert werden können.

Ist die Applikation installiert, soll sie, ebenfalls über den für den jeweiligen
Browser üblichen Weg, vollständig deinstalliert werden können.

Wird auf die oben erwähnte Ausweichmöglichkeit einer anderen Plattform
zurückgegriffen, wird durch Rückfragen bei den BetreiberInnen geprüft, ob dies
wirklich aufgrund von Einschränkungen oder Ausschluss seitens der BetreiberInnen
geschehen ist. Sollte das nicht der Fall (gewesen) sein, gilt diese Anforderung
als nicht erfüllt.

\subsubsection{Geltungsbereich}

Die Installierbarkeit hat keine Beziehung zu irgend einer funktionalen
Anforderung.

