\subsection{Verbrauchsverhalten}
\label{subsec:nichtfunktionale-verbrauchsverhalten}

Verbrauchsverhalten definiert nach \cite{iso:9126} die \enquote{Anzahl und Dauer
der benötigten Betriebsmittel bei der Erfüllung der Funktionen}.

\subsubsection{Beschreibung}

Da \textit{Google Muddle} für seine BenutzerInnen direkt funktional wenig
bietet, soll die Applikation auch möglichst unaufdringlich und
ressourcenschonend arbeiten. Es ist nicht sehr relevant, wann die einzelnen
Funktionen genau ausgeführt werden. Dies sollte daher möglichst dann geschehen,
wenn der Computer wenig zu tun hat - also etwa wenn niemand damit am arbeiten
ist.

\subsubsection{Messverfahren}

So viele BenutzerInnen wie möglich werden in einer Testphase mit einem Browser
arbeiten, ohne zu wissen, ob die Applikation installiert ist oder nicht. Dabei
dürfen sie keinen Unterschied feststellen. Die BenutzerInnen werden sowohl mit
aktueller, als auch mit alter Hardware arbeiten. Auch die Betriebssysteme und
die darauf installierten Browser werden divers sein (siehe Tabelle 
\ref{tbl:browser-gewichtung} auf Seite \pageref{tbl:browser-gewichtung}).

Wenn mehr als 5 Prozent der BenutzerInnen eindeutig erkennen, ob die Applikation
installiert ist oder nicht, gilt diese Anforderung als nicht erfüllt.

\subsubsection{Geltungsbereich}

Diese Anforderung gilt für die gesamten Laufzeit der Applikation. Als besonders
kritisch gilt dabei meist der Start, da in dieser Phase bewusst gewartet wird.
