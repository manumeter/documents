\section{Nichtfunktionale Anforderungen}
\label{sec:nichtfunktionale}

Eine Anforderung ist gemäss \cite{glinz:nfr} als nichtfunktional zu verstehen, 
wenn \enquote{das ihr zu Grunde liegende Bedürfnis eine nicht gegenständliche
Eigenschaft [der Applikation] ist}. In diesem Kapitel werden also Anforderungen
definiert, bei welchen das (der Anforderung zu Grunde liegende) Bedürfnis nicht
die Funktionalität selbst ist.

In Tabelle \ref{tbl:softwarequalitaet} ist die (vollständige) Liste der
Qualitätsmerkmale von Software gemäss \acs{ISO}/\acs{IEC} 9126 (\acs{DIN} 66272)
zu finden. Einige davon werden im Folgenden als nichtfunktionale Anforderungen
in Bezug auf \textit{Google Muddle} genauer beschrieben. Die übrigen werden im
Rahmen dieser Modularbeit nicht erfasst.
\\[\intextsep]
\begin{minipage}{\linewidth}
	\centering
	\begin{tabular}{llr}
		\hline
		Kategorie & Qualitätsmerkmal & Priorität \\
		\hline
		\multirow{6}{*}{Funktionalität:} & Angemessenheit & mittel \\
		 & \textbf{Richtigkeit} & \textbf{hoch} \\
		 & \textbf{Interoperabilität}
		  (\ref{subsec:nichtfunktionale-interoperabilitaet}) & \textbf{hoch} \\
		 & \textbf{Sicherheit} & \textbf{hoch} \\
		 & Ordnungsmässigkeit & mittel \\
		 & Konformität & gering \\
		\hline
		\multirow{4}{*}{Zuverlässigkeit:} & Reife & mittel \\
		 & \textbf{Fehlertoleranz} & \textbf{hoch} \\
		 & Wiederherstellbarkeit & gering \\
		 & Konformität & gering \\
		\hline
		\multirow{5}{*}{Benutzbarkeit:} & Verständlichkeit & mittel \\
		 & Erlernbarkeit & gering \\
		 & \textbf{Bedienbarkeit}
		  (\ref{subsec:nichtfunktionale-bedienbarkeit}) & \textbf{hoch} \\
		 & \textbf{Attraktivität} & \textbf{hoch} \\
		 & Konformität & gering \\
		\hline
		\multirow{3}{*}{Effizienz:} & Zeitverhalten & mittel \\
		 & \textbf{Verbrauchsverhalten}
		  (\ref{subsec:nichtfunktionale-verbrauchsverhalten}) & \textbf{hoch} \\
		 & Konformität & gering \\
		\hline
		\multirow{5}{*}{Wartbarkeit/Änderbarkeit:} & \textbf{Analysierbarkeit}
		 & \textbf{hoch} \\
		 & \textbf{Modifizierbarkeit} & \textbf{hoch} \\
		 & \textbf{Stabilität} & \textbf{hoch} \\
		 & Testbarkeit & mittel \\
		 & Konformität & gering \\
		\hline
		\multirow{5}{*}{Übertragbarkeit:} & \textbf{Anpassbarkeit}
		 & \textbf{hoch} \\
		 & \textbf{Installierbarkeit}
		  (\ref{subsec:nichtfunktionale-installierbarkeit}) & \textbf{hoch} \\
		 & Koexistenz & gering \\
		 & Austauschbarkeit & gering \\
		 & Konformität & gering \\
		\hline
	\end{tabular}
	\tabcaption{Qualitätsmerkmale nach \cite{iso:9126}}
	\label{tbl:softwarequalitaet}
\end{minipage}
\newpage

\subsection{Interoperabilität}
\label{subsec:nichtfunktionale-interoperabilitaet}

Interoperabilität bezeichnet gemäss \cite{iso:9126} die \enquote{Fähigkeit, mit
vorgegebenen Systemen zusammenzuwirken}.

\subsubsection{Beschreibung}

Es wird ein hoher Verbreitungsgrad der Applikation angestrebt, da sich dadurch
ihre Wirkung verbessert. Deshalb ist es wichtig, eine möglichst breite Variation
von Browsern zu unterstützen. Die Applikation soll mit den in der Tabelle
\ref{tbl:browser-gewichtung} aufgelisteten Browsern (gemäss Gewichtung) in
jeweils neuester Version (für Desktop-Betriebssysteme) kompatibel sein. Die
Gewichtung wurde auf Grundlage der Browserstatistik von
\cite{w3schools:browserstats} vorgenommen.
\\[\intextsep]
\begin{minipage}{\linewidth}
	\centering
	\begin{tabular}{llr}
		\hline
		Browser           & Betriebssysteme     & Gewichtung $g$ \\
		\hline
		Apple Safari      & Mac                 & \textbf{1} \\
		Google Chrome     & Linux, Mac, Windows & \textbf{4} \\
		Internet Explorer & Windows             & \textbf{2} \\
		Mozilla Firefox   & Linux, Mac, Windows & \textbf{3} \\
		Opera             & Mac, Windows        & \textbf{1} \\
		\hline
	\end{tabular}
	\tabcaption{Liste kompatibler Browser mit Gewichtung}
	\label{tbl:browser-gewichtung}
\end{minipage}

\subsubsection{Messverfahren}

Zwecks Messbarkeit wurden drei Stufen der Erfüllung, die pro Browser ermittelt
werden, definiert (siehe Tabelle \ref{tbl:interoperabilitaet-erfuellung}).
\\[\intextsep]
\begin{minipage}{\linewidth}
	\centering
	\begin{tabular}{ll}
		\hline
		Erfüllungsgrad $e$ & Beschreibung \\
		\hline
		\textbf{1}
		 & Die Applikation läuft mit diesem Browser. \\
		\multirow{2}{*}{\textbf{2}}
		 & Die Applikation läuft mit diesem Browser und \\
		 & bietet einen reduzierten Funktionsumfang. \\
		\multirow{2}{*}{\textbf{3}}
		 & Die Applikation läuft mit diesem Browser und \\
		 & bietet den vollen Funktionsumfang. \\
		\hline
	\end{tabular}
	\tabcaption{Erfüllungsgrad der Interoperabilität pro Browser}
	\label{tbl:interoperabilitaet-erfuellung}
\end{minipage}
\\[\intextsep]
Damit die Anforderung als erfüllt betrachtet wird, ist eine Summe für alle
Browser aus Tabelle \ref{tbl:browser-gewichtung} aus Gewichtung $g$
multipliziert mit Erfüllungsgrad $e$ (also $ \sum g * e $) von mindestens 29
erforderlich. Maximal ist 33 möglich.

\subsubsection{Geltungsbereich}

Diese Anforderung bezieht sich auf alle funktionalen Anforderungen, falls der
Erfüllungsgrad 3 ist. Ansonsten (Erfüllungsgrad 1 und 2) kann die Untermenge
beliebig ausgewählt werden (siehe Tabelle
\ref{tbl:interoperabilitaet-erfuellung}).

\newpage
\subsection{Bedienbarkeit}
\label{subsec:nichtfunktionale-bedienbarkeit}

Gemeint ist gemäss \cite{iso:9126} der \enquote{Aufwand für [...]
Benutzer[Innen], die Anwendung zu bedienen}.

\subsubsection{Beschreibung}

Da \textit{Google Muddle} vollständig automatisch funktionieren muss, ist keine
Interatktion mit BenutzerInnen gewünscht. Sie sollen auch nicht die Möglichkeit
haben, Konfigurationen vorzunehmen, da die Verhaltensweise der Applikation für
jedeN identisch sein soll. Dies vereinfacht die Weiterentwicklung vom Verhalten
aufgrund neuer Erkentnisse, da keine BenutzerInnenspezifische Einstellungen
berücksichtigt werden müssen. So können auch jeweils die optimalen Einstellungen
zur bestmöglichen Erreichung der Ziele (vgl. Kapitel
\ref{subsec:einleitun-motivation}) für alle gesetzt werden. Auch spezielle
Funktionen wie Updates sollen automatisch - also ohne BenutzerInneninteraktion
- stattfinden.

\subsubsection{Messverfahren}

So viele BenutzerInnen wie möglich werden die Applikation installiert. Nach der
Installation versuchen sie, irgendwie mit der Applikation zu interagieren. Dies
darf weder beim Start noch beim Beenden des Browsers möglich sein. Auch während
der gesamten Laufzeit des Browsers oder bei einem allfälligen speziellen
Funktion wie zum Beispiel einem Update darf die Erweiterung nicht wahrgenommen
werden. (Als Abgrenzung zum Verbrauchsverhalten (vgl. Kapitel 
\ref{subsec:nichtfunktionale-verbrauchsverhalten}), wo sie aufgrund des
Ressourcenverbrauchs nicht wahrgenommen werden darf, gilt hier einzig die
Wahrnehmung aufgrund von BenutzerInneninteraktionen wie zum Beispiel
Konfiguration oder Fehlermeldungen.)

\subsubsection{Geltungsbereich}

Diese Anforderung gilt für die gesamten Laufzeit der Applikation.

\newpage
\subsection{Verbrauchsverhalten}
\label{subsec:nichtfunktionale-verbrauchsverhalten}

Verbrauchsverhalten definiert nach \cite{iso:9126} die \enquote{Anzahl und Dauer
der benötigten Betriebsmittel bei der Erfüllung der Funktionen}.

\subsubsection{Beschreibung}

Da \textit{Google Muddle} für seine BenutzerInnen direkt funktional wenig
bietet, soll die Applikation auch möglichst unaufdringlich und
ressourcenschonend arbeiten. Es ist nicht sehr relevant, wann die einzelnen
Funktionen genau ausgeführt werden. Dies sollte daher möglichst dann geschehen,
wenn der Computer wenig zu tun hat - also etwa wenn niemand damit am arbeiten
ist.

\subsubsection{Messverfahren}

So viele BenutzerInnen wie möglich werden in einer Testphase mit einem Browser
arbeiten, ohne zu wissen, ob die Applikation installiert ist oder nicht. Dabei
dürfen sie keinen Unterschied feststellen. Die BenutzerInnen werden sowohl mit
aktueller, als auch mit alter Hardware arbeiten. Auch die Betriebssysteme und
die darauf installierten Browser werden divers sein (siehe Tabelle 
\ref{tbl:browser-gewichtung} auf Seite \pageref{tbl:browser-gewichtung}).

Wenn mehr als 5 Prozent der BenutzerInnen eindeutig erkennen, ob die Applikation
installiert ist oder nicht, gilt diese Anforderung als nicht erfüllt.

\subsubsection{Geltungsbereich}

Diese Anforderung gilt für die gesamten Laufzeit der Applikation. Als besonders
kritisch gilt dabei meist der Start, da in dieser Phase bewusst gewartet wird.

\newpage
\subsection{Installierbarkeit}
\label{subsec:nichtfunktionale-installierbarkeit}

\cite{iso:9126} definiert die Installierbarkeit als \enquote{Aufwand, der zum
Installieren der Software in einer festgelegten Umgebung notwendig ist}.

\subsubsection{Beschreibung}

Die Applikation soll im für den jweiligen Browser offiziellen Store für
Erweiterungen verfügbar sein.

Ist dies aufgrund von Einschränkungen oder Ausschluss seitens der BetreiberInnen
dieses Stores nicht möglich, soll die Applikation über andere Plattformen zum
herunterladen angeboten werden. Falls auf diese Möglichkeit zurückgegriffen
werden muss, soll auf der Webseite dieser Plattform zusätzlich zur Applikation
eine einfache Installationsanleitung präsentiert werden.

\subsubsection{Messverfahren}

\textit{Google Muddle} muss - in jedem der Browser aus Tabelle
\ref{tbl:browser-gewichtung} auf Seite \pageref{tbl:browser-gewichtung} - über
eine via Menü erreichbare Suchfunktion für Erweiterungen gefunden werden können.
Anschliessend soll sie, über den für den jeweiligen Browser üblichen Weg,
installiert werden können.

Ist die Applikation installiert, soll sie, ebenfalls über den für den jeweiligen
Browser üblichen Weg, vollständig deinstalliert werden können.

Wird auf die oben erwähnte Ausweichmöglichkeit einer anderen Plattform
zurückgegriffen, wird durch Rückfragen bei den BetreiberInnen geprüft, ob dies
wirklich aufgrund von Einschränkungen oder Ausschluss seitens der BetreiberInnen
geschehen ist. Sollte das nicht der Fall (gewesen) sein, gilt diese Anforderung
als nicht erfüllt.

\subsubsection{Geltungsbereich}

Die Installierbarkeit hat keine Beziehung zu irgend einer funktionalen
Anforderung.


