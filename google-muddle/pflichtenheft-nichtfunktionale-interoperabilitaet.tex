\subsection{Interoperabilität}
\label{subsec:nichtfunktionale-interoperabilitaet}

Interoperabilität bezeichnet gemäss \cite{iso:9126} die \enquote{Fähigkeit, mit
vorgegebenen Systemen zusammenzuwirken}.

\subsubsection{Beschreibung}

Es wird ein hoher Verbreitungsgrad der Applikation angestrebt, da sich dadurch
ihre Wirkung verbessert. Deshalb ist es wichtig, eine möglichst breite Variation
von Browsern zu unterstützen. Die Applikation soll mit den in der Tabelle
\ref{tbl:browser-gewichtung} aufgelisteten Browsern (gemäss Gewichtung) in
jeweils neuester Version (für Desktop-Betriebssysteme) kompatibel sein. Die
Gewichtung wurde auf Grundlage der Browserstatistik von
\cite{w3schools:browserstats} vorgenommen.
\\[\intextsep]
\begin{minipage}{\linewidth}
	\centering
	\begin{tabular}{llr}
		\hline
		Browser           & Betriebssysteme     & Gewichtung $g$ \\
		\hline
		Apple Safari      & Mac                 & \textbf{1} \\
		Google Chrome     & Linux, Mac, Windows & \textbf{4} \\
		Internet Explorer & Windows             & \textbf{2} \\
		Mozilla Firefox   & Linux, Mac, Windows & \textbf{3} \\
		Opera             & Mac, Windows        & \textbf{1} \\
		\hline
	\end{tabular}
	\tabcaption{Liste kompatibler Browser mit Gewichtung}
	\label{tbl:browser-gewichtung}
\end{minipage}

\subsubsection{Messverfahren}

Zwecks Messbarkeit wurden drei Stufen der Erfüllung, die pro Browser ermittelt
werden, definiert (siehe Tabelle \ref{tbl:interoperabilitaet-erfuellung}).
\\[\intextsep]
\begin{minipage}{\linewidth}
	\centering
	\begin{tabular}{ll}
		\hline
		Erfüllungsgrad $e$ & Beschreibung \\
		\hline
		\textbf{1}
		 & Die Applikation läuft mit diesem Browser. \\
		\multirow{2}{*}{\textbf{2}}
		 & Die Applikation läuft mit diesem Browser und \\
		 & bietet einen reduzierten Funktionsumfang. \\
		\multirow{2}{*}{\textbf{3}}
		 & Die Applikation läuft mit diesem Browser und \\
		 & bietet den vollen Funktionsumfang. \\
		\hline
	\end{tabular}
	\tabcaption{Erfüllungsgrad der Interoperabilität pro Browser}
	\label{tbl:interoperabilitaet-erfuellung}
\end{minipage}
\\[\intextsep]
Damit die Anforderung als erfüllt betrachtet wird, ist eine Summe für alle
Browser aus Tabelle \ref{tbl:browser-gewichtung} aus Gewichtung $g$
multipliziert mit Erfüllungsgrad $e$ (also $ \sum g * e $) von mindestens 29
erforderlich. Maximal ist 33 möglich.

\subsubsection{Geltungsbereich}

Diese Anforderung bezieht sich auf alle funktionalen Anforderungen, falls der
Erfüllungsgrad 3 ist. Ansonsten (Erfüllungsgrad 1 und 2) kann die Untermenge
beliebig ausgewählt werden (siehe Tabelle
\ref{tbl:interoperabilitaet-erfuellung}).
