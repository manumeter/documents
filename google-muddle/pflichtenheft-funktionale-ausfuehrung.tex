\subsection{Eruieren der Ausführung}
\label{subsec:funktionale-ausfuehrung}

Mit \enquote{Eruieren der Ausführung} ist gemeint, dass die Applikation ständig
überprüft, wie stark der Computer und die Netzwerkverbindung gerade ausgelastet
sind und, wenn genügend Ressourcen zur Verfügung stehen, Suchanfragen oder
Updates initiiert.

\subsubsection{Auslöser}

Auslöser ist der Start der Applikation (funktionale Anforderung
\ref{subsec:funktionale-start}).

\subsubsection{AkteurInnen}

AkteurInnen sind für diese funktionale Anforderung keine vorhanden.

\subsubsection{Essenzielle Schritte}

\begin{enumerate}
\item Die Applikation wartet eine zufällige Zeit zwischen 5 Sekunden und 2
Minuten\footnote[1]{Bei diesen Zeiten handelt es sich um Vorschläge, die im
Verlauf der Entwicklung durch Erfahrungswerte ersetzt werden sollten.}.
\item \textit{Google Muddle} überprüft, ob der Computer und die
Netzwerkverbindung ausgelastet sind. Falls eine Schwelle überschritten wird, die
für den/die BenutzerIn bemerkbar ist, wird Punkt 1 erneut ausgeführt. Um hier
eine Endlosschleife zu vermeiden, soll die Schwelle dynamisch an die
Gegebenheiten angepasst werden.
\item Falls die letzte Überprüfung schon über 7 Tage\footnote[2]{Die 7 Tage
werden als angemessen einschätzt. Auch hier darf die Zahl im Verlauf der
Entwicklung gerne angepasst werden, wenn die Erfahrung zeigt, dass ein anderer
Wert optimal wäre.} her ist überprüft \textit{Google Muddle}, ob eine neuere
Version verfügbar ist und installiert diese gegebenenfalls (funktionale
Anforderung \ref{subsec:funktionale-update}). Dabei wird der Zeitpunkt dieser
Überprüfung protokolliert. Sollte eine neuere Version installiert worden sein,
wird \textit{Google Muddle} neu gestartet.
\item Eine Suchanfrage wird gestartet (funktionale Anforderung
\ref{subsec:funktionale-suche}).
\item Die Schritte werden erneut ab Punkt 1 ausgeführt.
\end{enumerate}
