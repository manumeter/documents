\section{Einleitung}
\label{sec:einleitung}

\subsection{Zum Dokument}

Dieses Dokument soll als möglichst präzise Vorlage für die
Grobspezifikation der Applikation \textit{Google Muddle} dienen.

Neben den funktionalen (Kapitel \ref{sec:funktionale}) und den nichtfunktionalen
Anforderungen (Kapitel \ref{sec:nichtfunktionale}) werden in diesem Dokument
auch die Rahmenbedingungen (Kapitel \ref{sec:rahmenbedingungen}) festgelegt.
Ergänzend sind im Kapitel \ref{sec:einleitung} die Motivation sowie der
angestrebte Soll-Zustand kurz skizziert.

Am Ende des Dokuments können dann die Verweise auf zitierte Literatur, das
Abkürzungens- und das Tabellenverzeichnis gefunden werden.

Um die Geschlechtsneutralität der Aussagen zu gewährleisten werden in diesem
Dokument in der Regel Binnenmajuskeln verwendet. So sollen die weibliche und
die männliche Bezeichnung in kurzer Form vereint werden (vgl.
\url{https://de.wikipedia.org/wiki/Binnen-I}).

\subsection{Referenzdokumente}

Es sind keine Referenzdokumente vorhanden.

\subsection{Motivation}
\label{subsec:einleitun-motivation}

Wie die Enthüllungen durch \cite{snowden:nsa} gezeigt haben, werden (durch die
\acs{NSA}) flächendeckend Überwachungstechniken eingesetzt, die das Verhalten
fast aller Internetbesucher protokollieren und automatisch auswerten.
Multinationale Internetkonzerne wie Google mit strategisch gut geeignetem
Dienstleistungsangebot und ausreichendem Kundenstamm spielen dabei eine zentrale
Rolle. Um dieser golbalen Überwachung zu entgehen, oder sie wenigstens zu
erschweren, gibt es meines Erachtens drei Strategien, deren Kombination
höchstmöglichen Schutz der Privatsphäre bietet.

Erstens kann versucht werden, Nutz- und Metadaten so zu verschlüsseln bzw. zu
verschleiern (zum Beispiel durch die Umleitung über viele kaskadierte Knoten),
dass deren Protokollierung nutzlos würde, da die Auswertung kaum mehr möglich
wäre. Da diese Strategie sehr aufwendig ist und dazu noch von einer grossen
Anzahl BenutzerInnen umgesetzt werden müsste, damit die jeweilige Kommunikation
geschützt wäre, ist ihre baldige Verbreitung trotz vorhandener Werkzeuge meiner
Einschätzung nach eher unwahrscheinlich.

Zweitens könnte der Versuch unternommen werden, die überwachenden Unternehmen
und die überwachten Internetknoten komplett zu meiden. Dies würde faktisch zum
totalen Verzicht auf das Internet führen, da praktisch der gesamte Datenverkehr
über wenige einzelne (überwachte) Knoten verläuft. Zudem werden die am stärksten
genutzten Dienstleistung von einigen wenigen Konzernen angeboten, die gemäss
\cite{snowden:nsa} oft ebenfalls zur (geheimen) Überwachtung verpflichtet sind.

Die dritte Strategie, auf welche ich in diesem Projekt zurückgreifen werde,
funktioniert komplett anders. Anstatt die Nutz- und Metadaten so zu
manipulieren, dass sie nicht mehr ausgewertet werden können oder auf grosse
Teile des Internets zu verzichten, wird mit dieser Strategie versucht, die
echten Daten durch grosse Mengen von künstlich erzeugten Daten zu erweitern, die
sich kaum von den Echten unterscheiden lassen. Durch die entstehende Datenflut
wird die Auswertung der Protokolle stark verfälscht und nutzlos.

\subsection{Soll-Zustand}

Um diese dritte Strategie umzusetzen, sind verschiedene Werkzeuge, die jeweils
auf das verwendete Kommunikationsmittel bzw. die Dienstleistung angepasst sind,
notwendig. Um einen grossen Bereich der Internetnutzung abzudecken, soll in
diesem Projekt eine Erweiterung für Internet-Browser geschrieben werden, welche
die Google Suche wie folgt nutzt.

In der gesamten Zeit, in der ein Browser mit dieser Erweiterung läuft, sollen im
Hintergrund (ohne, dass der/die BenutzerIn dadurch gestört wird) Suchanfragen an
Google gestellt werden. Die Suchanfragen sowie die zeitlichen Abstände zwischen
einzelnen Anfragen sollen dabei möglichst nicht von echten Suchanfragen
unterschieden werden können. Anschliessend soll ein zufällig ausgewähltes
Suchresultat geöffnet werden. Als Quelle für Suchbegriffe könnten zum Beispiel
echte Suchanfragen anderer BenutzerInnen der Browser-Erweiterung dienen.

Diese Methode hat ausserdem den positiven Nebeneffekt, die so genannte
Filterblase zu verhindern. Unter Filterblase wird der Mechanismus verstanden,
dass Suchmaschinen BenutzerInnenprofile anlegen und den jeweiligen BenutzerInnen
auf sie zugeschnittene Ergebnisse anzeigen. Dabei entgehen den Leuten wichtige
Alternativen, die vom Algorithmus bereis ausgefiltert wurden.

Dabei soll sich diese Arbeit ausschliesslich mit dem Client-Teil der Software
beschäftigen. Allfällige Server-Komponenten sollen vorerst nicht berücksichtigt
werden.
